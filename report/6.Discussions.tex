\section{Discussion}

From section \ref{sec:results} it has been established that it was possible to develop a functioning 3D scanner which were able to create a three dimensional reconstruction of a rotating object using structured light. It has also been established, however, that the reconstruction method used has its flaws and that depending on the object, the optimal setup will be different. The point could reconstructions found in section \ref{sec:results} for both objects show large empty gaps due to occlusions and the reflective properties of the object surfaces. It could therefore be argued that the experimental setup should have been altered such that the camera was placed closer to the structured light source.\\

Another point of discussion is the calibration and computed focal length. Since it was decided that video footage was to be used for the reconstruction instead of a series of single photos, the images used for the calibration had to be the same resolution I.E 1920x1080. This yielded the focal length in pixels along $x$ and $y$ seen in camera matrix in section \ref{result:cameracalib}. In order to compute the focal length, the inverse of the pixel size is also needed. Here it was assumed that when the camera records video footage, only a section of the camera sensor is in use such that the pixel size is the same as when using the whole camera sensor. Using these assumptions yields a focal length of about 17-18mm as stated by equation \ref{eq:fxfy} which turned out gave the right scaling for the depth. The focal length when taking a photo is, however, also stored by the camera inside the image details but using the same focus and zoom yielded a focal length of 55mm. It is suspected that the computed focal length of about 17-18mm gives the correct depth scaling because the images are lower than maximum resolution. It could also have been that the calibration tool used in this report scales the camera matrix based on the resolution. In hindsight, the right approach would have been to use individual photos for the depth reconstruction such that both the focal length and the depth scaling would be correct.\\

Lastly, the performance of the 3D scanner could have been improved by more accurate positional measurements of the camera, object and light source. In this report, the distances was measured using a simple tape measure. Ideally, a more reliable and precise measuring approach should have, which would have yielded a better 3D reconstruction.   


