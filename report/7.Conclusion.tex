\newpage
\section{Conclusion}

In conclusion it has been shown that the overall objective of developing a functional 3D scanner has been fulfilled. Using knowledge of camera models and trigonometry a mathematical was first developed which was able to estimate the depth of an object by relating how a series of vertical structured light beams is deformed over the surface of the object. Using knowledge of camera calibration it was possible to scale the found depth estimates to the correct real world values. Furthermore, correcting of unwanted radial distortion caused by the lenses inside the camera was also facilitated. \\


By rotating the object a full 360 degrees it was then possible to obtain a full three dimensional reconstruction. After analysing the found mathematical model, it was possible to derive an upper bound for the expected depth precision and it was furthermore established that the location of the camera, object and light source has a large impact on this maximum theoretical depth precision. It was also found, however, that depending on the maximum depth variation of object, the optimal setup will be different.\\

Using several different methods of image analysis like erosion, HSV-thresholding, contouring algorithms and epipolar geometry, it was possible to detect the location of the structured light beams on the rotating object. In this regard, the effectiveness of each of these methods was assessed before finally the most reliable method was found. Despite several tests, however, it was concluded that depending on the lighting conditions and the reflective properties of the object the optimal detection method will be different.\\

